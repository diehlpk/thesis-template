% Dieses Dokument muss mit PDFLatex geestzt werden
% Vorteil: Grafiken koennen als jpg, png, ... verwendet werden
%          und die Links im Dokument sind auch gleich richtig
%
%Ermöglicht \\ bei der Titelseite (z.B. bei supervisor)
%Siehe https://github.com/latextemplates/uni-stuttgart-cs-cover/issues/4
\RequirePackage{kvoptions-patch}
%
\documentclass[
               paper=a4,
%               twoside, % fuer die Betrachtung am Schirm ungeschickt
% Optionen fuer typearea.
               BCOR1.92mm,DIV12,headinclude, %je höher der DIV-Wert, desto mehr geht auf eine Seite - Hack für BCOR. Bei BCOR2mm sind die Fuellpunkte beim Inhaltsverzeichnis falsch
%               titlepage,
               bibliography=totoc,
%               idxtotoc,   %Index ins Inhaltsverzeichnis
%				liststotoc, %List of X ins Inhaltsverzeichnis, mit liststotocnumbered werden die Abbildungsverzeichnisse nummeriert
               headsepline,
               cleardoublepage=empty,
               parskip=half,
%				pointlessnumbers, %f"ur englische Texte, dann unten \ifdeutsch und \ifenglisch anpassen.
%               draft    % um zu sehen, wo noch nachgebessert werden muss - wichtig, da Bindungskorrektur mit drin
               final   % ACHTUNG! - in pagestyle.tex noch Seitenstil anpassen
               ]{scrbook}

%Englisch:			   
%\let\ifdeutsch\iffalse
%\let\ifenglisch\iftrue

%Deutsch:
\let\ifdeutsch\iftrue
\let\ifenglisch\iffalse

			   
%%%
% Beschreibung:
% In dieser Datei werden zuerst die benoetigten Pakete eingebunden und
% danach diverse Optionen gesetzt. Achtung Reihenfolge ist entscheidend!
%
%%%


%%%
% Styleguide:
%
% Ein sehr kleiner Styleguide. Packages werden in Blöcken organisiert.
% Ein Block beginnt mit drei % in einer Zeile, dann % <Blocküberschrift>, dann 
% eine Liste der möglichen Optionen und deren Einstellungen, Gründe und Kommentare
% eine % Zeile in der sonst nichts steht und dann wieder %%% in einer Zeile.
%
% Zwischen zwei Blöcken sind 2 Leerzeilen!
% Zu jedem Paket werden soviele Optionen wie möglich/nötig angegeben
%
%%%

%%%
% Codierung
% Wir sind im 21 Jahrhundert, utf-8 löst so viele Probleme.
%
% Mit UTF-8 funktionieren folgende Pakete nicht mehr. Bitte beachten!
%   * fancyvrb mit § 
%   * easylist -> http://www.ctan.org/tex-archive/macros/latex/contrib/easylist/ 
\usepackage[utf8]{inputenc}
%
%%%

%%%
%Parallelbetrieb tex4ht und pdflatex
\makeatletter
\@ifpackageloaded{tex4ht}{\def\iftex4ht{\iftrue}}
                         {\def\iftex4ht{\iffalse}}
\makeatother
%%%


%%%
%Farbdefinitionen
\usepackage[hyperref,dvipsnames]{xcolor}
%

%%%
%Biblatex
\usepackage[style=alphabetic,backref=true,backend=biber]{biblatex}
%

%%%
% Neue deutsche Rechtschreibung und Literatur statt "Literature", Nachfolger von ngerman.sty
\ifdeutsch
\usepackage[ngerman]{babel}
  %Ein "abstract" ist eine "Kurzfassung", keine "Zusammenfassung"
  \addto\captionsngerman{%
    \renewcommand\abstractname{Kurzfassung}%
  }
\else
%
%
% if you are writing in english
% für englische Texte, Hinweise zu weiteren, notwendigen Umstellungen in README.txt beachten
\usepackage[american]{babel}
\fi
%
%%%

%%%
% Usage of typesetting values with SI units.
\usepackage{siunitx}
%%%
%%%
% Anführungszeichen
% Zitate in \enquote{...} setzen, dann werden automatisch die richtigen Anführungszeichen verwendet.
\usepackage{csquotes}
%%%


%%%
% erweitertes Enumerate
\usepackage{paralist}
%
%%%


%%%
% fancyheadings (nicht nur) fuer koma
\usepackage[automark]{scrpage2} 
%
%%%


%%%
%Mathematik
%
\usepackage[fleqn,leqno]{amsmath} % Viele Mathematik-Sachen: Doku: /usr/share/doc/texmf/latex/amsmath/amsldoc.dvi.gz
%fleqn (=Gleichungen linksbündig platzieren) funktioniert nicht direkt. Es muss noch ein Patch gemacht werden:
\addtolength\mathindent{1em}%work-around ams-math problem with align and 9 -> 10
\usepackage{mathtools} %fixes bugs in AMS math
%
%for theorems, replacement for amsthm
\usepackage[amsmath,hyperref]{ntheorem}
\theorempreskipamount 2ex plus1ex minus0.5ex
\theorempostskipamount 2ex plus1ex minus0.5ex
\theoremstyle{break}
\newtheorem{definition}{Definition}[section]
%
%%%


%%%
% Intelligentes Leerzeichen um hinter Abkürzungen die richtigen Abstände zu erhalten, auch leere.
% siehe commands.tex \gq{}
\usepackage{xspace}
%
%%%


%%%
% Anhang
\usepackage{appendix}
%[toc,page,title,header]
%
%%%


%%%
% Grafikeinbindungen
\usepackage{graphicx}%Parameter "pdftex" unnoetig
\graphicspath{{\getgraphicspath}}
\newcommand{\getgraphicspath}{graphics/}
%
%%%


%%%
% Enables inclusion of SVG graphics - 1:1 approach
% This is NOT the approach of http://www.ctan.org/tex-archive/info/svg-inkscape,
% which allows text in SVG to be typeset using LaTeX
% We just include the SVG as is
\usepackage{epstopdf}
\epstopdfDeclareGraphicsRule{.svg}{pdf}{.pdf}{%
  inkscape -z -D --file=#1 --export-pdf=\OutputFile
}
%
%%%


%%%
% Enables inclusion of SVG graphics - text-rendered-with-LaTeX-approach
% This is the approach of http://www.ctan.org/tex-archive/info/svg-inkscape,
\newcommand{\executeiffilenewer}[3]{%
\IfFileExists{#2}
{
%\message{file #2 exists}
\ifnum\pdfstrcmp{\pdffilemoddate{#1}}%
{\pdffilemoddate{#2}}>0%
{\immediate\write18{#3}}
\else
{%\message{file up to date #2}
}
\fi%
}{
%\message{file #2 doesn't exist}
%\message{argument: #3}
%\immediate\write18{echo "test" > xoutput.txt}
\immediate\write18{#3}
}
}
\newcommand{\includesvg}[1]{%
\executeiffilenewer{#1.svg}{#1.pdf}%
{
inkscape -z -D --file=\getgraphicspath#1.svg %
--export-pdf=\getgraphicspath#1.pdf --export-latex}%
\input{\getgraphicspath#1.pdf_tex}%
}
%%%

%%%
% Tabellenerweiterungen
\usepackage{array} %increases tex's buffer size and enables ``>'' in tablespecs
\usepackage{longtable}
%
%%%

%%%
% Eine Zelle, die sich über mehrere Zeilen erstreckt.
% Siehe Beispieltabelle in Kapitel 2
\usepackage{multirow}
%
%%%


%%%
% Links verhalten sich so, wie sie sollen
\usepackage{url}
%
%%%


%%%
% Index über Begriffe, Abkürzungen
%\usepackage{makeidx} makeidx ist out -> http://xindy.sf.net verwenden
%
%%%

%%%
%lustiger Hack fuer das Abkuerzungsverzeichnis
%nach latex durchlauf folgendes ausfuehren
%makeindex ausarbeitung.nlo -s nomencl.ist -o ausarbeitung.nls 
%danach nochmal latex
%\usepackage{nomencl}
%	\let\abk\nomenclature %Deutsche Ueberschrift setzen
%	  	\renewcommand{\nomname}{List of Abbreviations}
%		%Punkte zw. Abkuerzung und Erklaerung
%	  	\setlength{\nomlabelwidth}{.2\hsize}
%	  	\renewcommand{\nomlabel}[1]{#1 \dotfill}
%		%Zeilenabstaende verkleinern
%	  	\setlength{\nomitemsep}{-\parsep}
%	\makenomenclature
%
%%%

%%%
% Logik für Tex
\usepackage{ifthen} %fuer if-then-else @ commands.tex
%
%%%


%%%
% unterschiedliche Fancy-Chapter-Styles
%\usepackage[Bjarne]{fncychap}
%\usepackage[Lenny]{fncychap}
%
%%%


%%%
%
\usepackage{listings}

\lstset{                                       
        language=C,                            
        basicstyle=\footnotesize\ttfamily,    
        %numbers=left,                         
        numberstyle=\tiny,                    
        %stepnumber=2,                        
        numbersep=10pt,                       
        tabsize=2,                            
        extendedchars=true,
        breaklines=true,                      
        captionpos=b,
        frame=trLB,
        keywordstyle=\color{blue},
        commentstyle=\color{gray},
        stringstyle=\ttfamily\color{red},
        showspaces=false,                     
        showtabs=false,                        
        xleftmargin=.1\textwidth,
        xrightmargin=.1\textwidth,
        framexleftmargin=17pt,
        framexrightmargin=5pt,
        framexbottommargin=4pt,
        framextopmargin=4pt,
        showstringspaces=false                
}

%
%%%


%%%
%Alternative zu Listings ist fancyvrb. Kann auch beides gleichzeitig benutzt werden.
\usepackage{fancyvrb}
%\fvset{fontsize=\small} %Groesse fuer den Fliesstext. Falls deaktiviert: \normalsize
%Funktioniert mit UTF-8 nicht mehr
%\DefineShortVerb{\§} %Somit kann im Text ganz einfach |verbatim| text gesetzt werden.
\RecustomVerbatimEnvironment{Verbatim}{Verbatim}{fontsize=\footnotesize}
\RecustomVerbatimCommand{\VerbatimInput}{VerbatimInput}{fontsize=\footnotesize}
%
%%%


%%%
% Bildunterschriften bei floats genauso formatieren wie bei Listings
% Anpassung wird unten bei den newfloat-Deklarationen vorgenommen
% Caption2 vielleicht besser
\usepackage{caption}
%
%%%


%%%
% Ermoeglicht es, Abbildungen um 90 Grad zu drehen
% Alternatives Paket: rotating Allerdings wird hier nur das Bild gedreht, während bei lscape auch die PDF-Seite gedreht wird. 
%Das Paket lscape dreht die Seite auch nicht 
\usepackage{pdflscape}
%
%%%


%%%
% Fuer listings
% Wird für fancyvrb und für lstlistings verwendet
% zustäzlich für den Paramter [H] = Floats WIRKLICH da wo sie deklariert wurden paltzieren - ganz ohne Kompromisse
% floatrow ist der Nachfolger von float
\iftex4ht
\usepackage{float}
\else
%tex4ht is not compatible with the advanced floatrow package
\usepackage{floatrow}
\fi
%
%%%


%%%
% Fuer Abbildungen innerhalb von Abbildungen
% Ersetzt das Paket subfigure
%\usepackage{subfig}
%
%%%


%%%
%Fuer Tabellen mit Variablen Spaltenbreiten
%\usepackage{tabularx}
%\usepackage{tabulary}
%
%%%


%%%
% Fußnoten
% 
%\usepackage{dblfnote}  %Zweispaltige Fußnoten
%
% Keine hochgestellten Ziffern in der Fußnote (KOMA-Script-spezifisch):
%\deffootnote[1.5em]{0pt}{1em}{\makebox[1.5em][l]{\bfseries\thefootnotemark}} 
%
% Abstand zwischen Fußnoten vergrößern:
%\setlength{\footnotesep}{.85\baselineskip}
%
%
\renewcommand{\footnoterule}{}             % Keine Trennlinie zur Fußnote 
\addtolength{\skip\footins}{\baselineskip} % Abstand Text <-> Fußnote
% Fußnoten immer ganz unten auf einer \raggedbottom-Seite
\usepackage{fnpos}
%
%%%


%%%
%
\raggedbottom     % Variable Seitenhöhen zulassen
%
%%%


%%%
% Falls die Seitenzahl bei einer Referenz auf eine Abbildung nur dann angegeben werden soll,
% falls sich die Abbildung nicht auf der selben Seite befindet...
\iftex4ht
%tex4ht does not work well with vref, therefore we emulate vref behavior
\newcommand{\vref}[1]{\ref{#1}}
\else
\ifdeutsch
\usepackage[ngerman]{varioref}
\else
\usepackage{varioref}
\fi
\fi
%%%

%%%
% Noch schoenere Tabellen als mit booktabs mit http://www.zvisionwelt.de/downloads.html
\usepackage{booktabs} 
%
%\usepackage[section]{placeins}
%
%%%


%%%
%Fuer Graphiken. Allerdings funktioniert es nicht zusammen mit pdflatex
%\usepackage{gastex} % \tolarance kann dann nicht mehr umdefiniert werden
%
%%%


%%%
%
%\usepackage{multicol}
%\usepackage{setspace} % kollidiert mit diplomarbeit.sty
%
%http://www.tex.ac.uk/cgi-bin/texfaq2html?label=floats
%\usepackage{flafter} %floats IMMER nach ihrer Deklaration platzieren
%
%%%


%%%
%schoene TODOs
\usepackage{todonotes}
\let\xtodo\todo
\renewcommand{\todo}[1]{\xtodo[inline,color=black!5]{#1}}
\newcommand{\utodo}[1]{\xtodo[inline,color=green!5]{#1}}
\newcommand{\itodo}[1]{\xtodo[inline]{#1}}
%
%%%


%%%
% Neue Pakete bitte VOR hyperref einbinden. Insbesondere bei Verwendung des
% Pakets "index" wichtig, da sonst die Referenzierung nicht funktioniert.
% Für die Indizierung selbst ist unter http://xindy.sourceforge.net
% ein gutes Tool zu erhalten 
%%%


%%%
%
% hier also neue packages einbinden
%
%%%


%%%
% ggf.in der Endversion komplett rausnehmen. dann auch \href in commands.tex aktivieren
% Alle Optionen nach \hypersetup verschoben, sonst crash
%
\usepackage[]{hyperref}%siehe auch: "Praktisches LaTeX" - www.itp.uni-hannover.de/~kreutzm
%
%% Da es mit KOMA 3 und xcolor zu Problemen mit den global Options kommt MÜSSEN die Optionen so gesetzt werden.
%

% Eigene Farbdefinitionen ohne die Namen des xcolor packages
\definecolor{darkblue}{rgb}{0,0,.5}
\definecolor{black}{rgb}{0,0,0}

\hypersetup{
	breaklinks=true,
	bookmarksnumbered=true,
	bookmarksopen=true,
	bookmarksopenlevel=1,
	breaklinks=true,
	colorlinks=true,
	pdfstartview=Fit,
	pdfpagelayout=SinglePage,
	%
	filecolor=darkblue,
	urlcolor=darkblue,
	linkcolor=black,
	citecolor=black
}
%
%%%


%%%
% cleveref für cref statt autoref, da cleveref auch bei Definitionen funktioniert
\ifdeutsch
\usepackage[ngerman,capitalise,nameinlink]{cleveref}
\else
\usepackage[capitalise,nameinlink]{cleveref}
\fi
%%%


%%%
% Zur Darstellung von Algorithmen
% Algorithm muss nach hyperref geladen werden
\usepackage[chapter]{algorithm} 
\usepackage[]{algpseudocode}
%
%%%


%%%
% Schriften
\input{preambel/fonts}
%
%%%


%%%
% Links auf Gleitumgebungen springen nicht zur Beschriftung,
% Doc: http://mirror.ctan.org/tex-archive/macros/latex/contrib/oberdiek/hypcap.pdf
% sondern zum Anfang der Gleitumgebung
\usepackage[all]{hypcap}
%%%


%%%
% Deckblattstyle
%
% TODO
%
%%%

%%%
% Rand
\input{preambel/margins}
%
%%%

%%%
% Optionen                                                                  
%
%Skip=0 funktioniert nicht
\captionsetup{format=hang,labelfont=bf,justification=justified,singlelinecheck=false,skip=0pt}
%
%neue float Umgebung fuer Listings, die mittels fancyvrb gesetzt werden sollen
\floatstyle{ruled}
\newfloat{Listing}{tbp}{code}[chapter]
\newfloat{Algorithmus}{tbp}{alg}[chapter]
%
%amsmath
%\numberwithin{equation}{section}
%\renewcommand{\theequation}{\thesection.\Roman{equation}}
%
%pdftex
\pdfcompresslevel=9
%
%Tabellen (array.sty)
\setlength{\extrarowheight}{1pt}
%
%
\input{preambel/chapterheads}
%
%%%


%%%
%Minitoc-Einstellungen
%\dominitoc
%\renewcommand{\mtctitle}{Inhaltsverzeichnis dieses Kapitels}
%
% Disable single lines at the start of a paragraph (Schusterjungen)
\clubpenalty = 10000
%
% Disable single lines at the end of a paragraph (Hurenkinder)
\widowpenalty = 10000 \displaywidowpenalty = 10000
%
%http://groups.google.de/group/de.comp.text.tex/browse_thread/thread/f97da71d90442816/f5da290593fd647e?lnk=st&q=tolerance+emergencystretch&rnum=5&hl=de#f5da290593fd647e
%Mehr Infos unter http://www.tex.ac.uk/cgi-bin/texfaq2html?label=overfull
\tolerance=2000
\setlength{\emergencystretch}{3pt}   % kann man evtl. auf 20 erhoehen
\setlength{\hfuzz}{1pt}
%
%%%


%%%
% Fuer listings.sty
\lstset{language=XML,
        showstringspaces=false,
        extendedchars=true,
        basicstyle=\footnotesize\ttfamily,
        commentstyle=\slshape,
        stringstyle=\ttfamily, %Original: \rmfamily, damit werden die Strings im Quellcode hervorgehoben		zusaetzlich evtl.: \scshape oder \rmfamily durch \ttfamily ersetzen. Dann sieht's aus, wie bei fancyvrb
        breaklines=true,
        breakatwhitespace=true,
        columns=flexible,
        aboveskip=0mm, %deaktivieren, falls man lstlistings direkt als floating object benutzt (\begin{lstlisting}[float,...])
        belowskip=0mm, %deaktivieren, falls man lstlistings direkt als floating object benutzt (\begin{lstlisting}[float,...])
        captionpos=b
}
\ifdeutsch
\renewcommand{\lstlistlistingname}{Verzeichnis der Listings}
\fi
%
%%%


%%%
%fuer algorithm.sty: - falls Deutsch und nicht Englisch. Falls Englisch als Sprache gewählt wurde, bitte die folgenden beiden Zeilen auskommentieren.
\floatname{algorithm}{Algorithmus}
\ifdeutsch
\renewcommand{\listalgorithmname}{Verzeichnis der Algorithmen}
\fi
%
%%%


%%%
% Das Euro Zeichen 
% Fuer Palatino (mathpazo.sty): richtiges Euro-Zeichen
% Alternative: \usepackage{eurosym}
\newcommand{\EUR}{\ppleuro}
%
%%%


%%%
%
% Float-placements - http://dcwww.camd.dtu.dk/~schiotz/comp/LatexTips/LatexTips.html#figplacement
% and http://people.cs.uu.nl/piet/floats/node1.html
\renewcommand{\topfraction}{0.85}
\renewcommand{\bottomfraction}{0.95}
\renewcommand{\textfraction}{0.1}
\renewcommand{\floatpagefraction}{0.75}
%\setcounter{totalnumber}{5}
%
%%%

%%%
%
% Bei Gleichungen nur dann die Nummer zeigen, wenn die Gleichung auch referenziert wird
%
% Funktioniert mit MiKTeX Stand 2012-01-13 nicht. Deshalb ist dieser Schalter deaktiviert.
%
%\mathtoolsset{showonlyrefs}
%
%%%

%%%
%Optischer Randausgleich
\usepackage{microtype}
%%%

%%%
% tikz (optional if the ppackage is installed)
\IfFileExists{tikz.sty}{
\usepackage{tikz}
}{}
%%%%

%%%
% pgfplots (optional if the ppackage is installed)
\IfFileExists{pgfplots.sty}{
\usepackage{pgfplots}
}{}
%%%%




%\usepackage{BA_Titelseite}

%Name of the authors
\author{Surname(s) Names(s)}

%Birthdday of the author
\geburtsdatum{1. April 1900}
%Birthday of the author
\geburtsort{Bonn}
%Submission date
\date{\today}

%Name of the supervisor
\betreuer{Betreuer: Prof. Dr. X Y}
\zweitgutachter{Zweitgutachter: Prof. Dr. Z} 
%Name of the department
%z.B.: Mathematisches Institut
\institut{XX}
%Title
\title{This is only an example}
%Do not change!
\ausarbeitungstyp{Bachelorarbeit Mathematik}


%Name of the author
\author{Surname Name}

%Submission date
\date{\today}

%Title
\title{Example title of this work}

 %Der untere Rand darf "flattern"
\raggedbottom

%%%
% Wie tief wird das Inhaltsverzeichnis aufgeschlüsselt
% 0 --\chapter
% 1 --\section % fuer kuerzeres Inhaltsverzeichnis verwenden - oder minitoc benutzen
% 2 --\subsection
% 3 --\subsubsection
% 4 --\paragraph
\setcounter{tocdepth}{1}
%
%%%

\makeindex

%Angaben in die PDF-Infos uebernehmen
\makeatletter
\hypersetup{
            pdftitle={}, %Titel der Arbeit
            pdfauthor={}, %Author
            pdfkeywords={}, % CR-Klassifikation und ggf. weitere Stichworte
            pdfsubject={}
}
\makeatother

\begin{document}

\maketitle

%tex4ht-Konvertierung verschönern
\iftex4ht
% tell tex4ht to create picures also for formulas starting with '$'
% WARNING: a tex4ht run now takes forever!
\Configure{$}{\PicMath}{\EndPicMath}{} 
%$ % <- syntax highlighting fix for emacs
\Css{body {text-align:justify;}}

%conversion of .pdf to .png
\Configure{graphics*}  
         {pdf}  
         {\Needs{"convert \csname Gin@base\endcsname.pdf  
                               \csname Gin@base\endcsname.png"}%  
          \Picture[pict]{\csname Gin@base\endcsname.png}%  
         }  
\fi

%Tipp von http://goemonx.blogspot.de/2012/01/pdflatex-ligaturen-und-copynpaste.html
%siehe auch http://tex.stackexchange.com/questions/4397/make-ligatures-in-linux-libertine-copyable-and-searchable
%
%ONLY WORKS ON MiKTeX
%On other systems, download glyphtounicode.tex from http://pdftex.sarovar.org/misc/
%
\input glyphtounicode.tex
\pdfgentounicode=1

\VerbatimFootnotes %verbatim text in Fußnoten erlauben. Geht normalerweise nicht.
%\frontmatter
\input{macros/commands}
\pagenumbering{arabic}

%TODO: \Titelblatt

%Eigener Seitenstil fuer die Kurzfassung und das Inhaltsverzeichnis
\deftripstyle{preamble}{}{}{}{}{}{\pagemark}
%Doku zu deftripstyle: scrguide.pdf
\pagestyle{preamble}
\renewcommand*{\chapterpagestyle}{preamble}

%Kurzfassung / abstract
%auch im Stil vom Inhaltsverzeichnis
\ifdeutsch
\section*{Kurzfassung}
\else
\section*{Abstract}
\fi
\ldots ... Short summary of the thesis ...
\cleardoublepage


% BEGIN: Verzeichnisse

\iftex4ht
\else
\microtypesetup{protrusion=false}
\fi

%%%
% Literaturverzeichnis ins TOC mit aufnehmen, aber nur wenn nichts anderes mehr hilft!
% \addcontentsline{toc}{chapter}{Literaturverzeichnis}
%
% oder zB
%\addcontentsline{toc}{section}{Abkürzungsverzeichnis}
%\section*{Abkürzungsverzeichnis}
%
%%%

%Inhaltsverzeichnis anlegen
\tableofcontents

% Bei einem ungünstigen Seitenumbruch im Inhaltsverzeichnis, kann dieser mit
% \addtocontents{toc}{\protect\newpage}
% an der passenden Stelle im Fließtext erzwungen werden.

%listof* untereinandergesetzt
%ACHTUNG! Falls ein anderer Kapitelstil gewählt wird, muss der Code hier evtl.
%  angepasst werden
\begingroup 
\makeatletter
  \def\@makeschapterhead#1{%
  \vspace*{10\p@}%
  {\parindent \z@ \raggedright \reset@font
            \normalfont \vphantom{\@chapapp{} \thechapter}
        \par\nobreak\vspace*{10\p@}%
        \interlinepenalty\@M
    {\huge \bfseries %
	%
	%Default-Schrift: Serifenhaft (fuer englische Dokumente)
	% Dann sowohl A als auch B deaktivieren
    %A) Fuer serifenlose Schrift folgende Zeile aktivieren:
	\ifdeutsch
    \fontfamily{phv}\selectfont
	\fi
	%B) Fuer Kapitaelchen folgende Zeile aktivieren:
	%\fontseries{m}\fontshape{sc}\selectfont
	%
	#1\par\nobreak}
    %\vspace*{1\p@}%
\makebox[\textwidth]{\hrulefill}%    \hrulefill alone does not work
    \par\nobreak
    \vskip 5\p@
  }}
\makeatother
\let\cleardoublepage\clearpage
\listoffigures
\let\cleardoublepage\relax
\listoftables

%Wird nur bei Verwendung von der lstlisting-Umgebung mit dem "caption"-Parameter benoetigt
%\lstlistoflistings 
%ansonsten:
\ifdeutsch
\listof{Listing}{Verzeichnis der Listings}
\else
\listof{Listing}{List of Listings}
\fi

%mittels \newfloat wurde die Algorithmus-Gleitumgebung definiert.
%Mit folgendem Befehl werden alle floats dieses Typs ausgegeben
\ifdeutsch
\listof{Algorithmus}{Verzeichnis der Algorithmen}
\else
\listof{Algorithmus}{List of Algorithms}
\fi
%\listofalgorithms %Ist nur für Algorithmen, die mittels \begin{algorithm} umschlossen werden, nötig

\endgroup

\cleardoublepage

\iftex4ht
\else
%Optischen Randausgleich und Grauwertkorrektur wieder aktivieren
\microtypesetup{protrusion=true}
\fi

% END: Verzeichnisse


\renewcommand*{\chapterpagestyle}{scrplain}
\pagestyle{scrheadings}
\input{preambel/pagestyle}
%
%
% ** Hier wird der Text eingebunden **
%
\input{content/einleitung}
%\input{...weitere Kapitel...}
%Die Angabe des schlauen Spruchs auf diesem Wege funtioniert nur,
%wenn keine Änderung des Kapitels mittels den in preambel/chapterheads.tex
%vorgeschlagenen Möglichkeiten durchgeführt wurde.
\setchapterpreamble[u]{%
\dictum[Albert Einstein]{Probleme kann man niemals mit derselben Denkweise lösen, durch die sie entstanden sind.}
}
\chapter{Kapitel zwei}
\label{chap:k2}
%\vspace{-3cm}
%\vspace{2cm}

Hier wird der Hauptteil stehen. Falls mehrere Kapitel gewünscht, entweder mehrmals \texttt{\textbackslash{}chapter} benutzen oder pro Kapitel eine eigene Datei anlegen und \texttt{ausarbeitung.tex} anpassen.

\section{File-Encoding und Unterstützung von Umlauten}
Die Vorlage wurde 2010 auf UTF-8 umgestellt.
TeXnicCenter 1 RC 1 unterstützt \textbf{kein} UTF-8.
Die Alpha-Version soll UTF-8 können, aber es gibt anscheinend Probleme.
Deshalb bitte einen anderen Editor, wie \zB \href{http://texstudio.sourceforge.net/}{TeXstudio} verwenden.

\section{Zitate}
Wörter am besten mittels \texttt{\textbackslash enquote\{...\}} \enquote{einschließen}, dann werden die richtigen Anführungszeichen verwendet.

\section{Mathematische Formeln}
\label{sec:mf}
Mathematische Formeln kann man $so$ setzen. \texttt{symbols-a4.pdf} (zu finden auf \url{http://www.ctan.org/tex-archive/info/symbols/comprehensive/symbols-a4.pdf}) enthält eine Liste der unter LaTeX direkt verfügbaren Symbole. z.\,B.\ $\mathbb{N}$ für die Menge der natürlichen Zahlen. Für eine vollständige Dokumentation für mathematischen Formelsatz sollte die Dokumentation zu \texttt{amsmath}, \url{ftp://ftp.ams.org/pub/tex/doc/amsmath/} gelesen werden.

Folgende Gleichung erhält keine Nummer, da \texttt{\textbackslash equation*} verwendet wurde.
\begin{equation*}
x = y
\end{equation*}

Die Gleichung~\ref{eq:test} erhält eine Nummer:
\begin{equation}
\label{eq:test}
x = y
\end{equation}

Eine ausführliche Anleitung zum Mathematikmodus von LaTeX findet sich in \url{http://www.ctan.org/tex-archive/help/Catalogue/entries/voss-mathmode.html}.

\section{Abbildungen}
Die Abbildungen~\ref{fig:chor1} und~\ref{fig:chor2} sind für das Verständnis dieses Dokuments
wichtig. Im Anhang zeigt Abbildung~\vref{fig:AnhangsChor} erneut die komplette Choreographie.

%Die Parameter in eckigen Klammern sind optionale Parameter - z.B. [htb!]
%htb! bedeutet: "Liebes LaTeX, bitte platziere diese Abbildung zuerst hier ("_h_ere"). Falls das nicht funktioniert, dann bitte oben auf der Seite ("_t_op"). Und falls das nicht geht, bitte unten auf der Seite ("_b_ottom"). Und bitte, bitte bevorzuge hier und oben, auch wenn's net so optimal aussieht ("!")
%Diese sollten nach Möglichkeit NICHT verwendet werden. LaTeX's Algorithmus für das Platzieren der Gleitumgebung ist schon sehr gut!
\begin{figure}
  \begin{center}
    \includegraphics[width=\textwidth]{choreography.pdf}
    \caption{Beispiel-Choreographie}
    \label{fig:chor1}
  \end{center}
\end{figure}

\begin{figure}
  \begin{center}
    \includegraphics[width=.8\textwidth]{choreography.pdf}
    \caption[Beispiel-Choreographie]{Die Beispiel-Choreographie. Nun etwas kleiner, damit \texttt{\textbackslash textwidth} demonstriert wird. Und auch die Verwendung von alternativen Bildunterschriften für das Verzeichnis der Abbildungen. Letzteres ist allerdings nur Bedingt zu empfehlen, denn wer liesst schon so viel Text unter einem Bild? Oder ist es einfach nur Stilsache?}
    \label{fig:chor2}
  \end{center}
\end{figure}

Das SVG in \cref{fig:directsvg} ist direkt eingebunden, während der Text im SVG in \cref{fig:latexsvg} mittels pdflatex gesetzt ist.
\todo{Falls man die Graphiken sehen möchte, muss inkscape im PATH sein und im Tex-Quelltext \texttt{\textbackslash{}iffalse} und \texttt{\textbackslash{}iftrue} auskommentiert sein.}

\iffalse % <-- Das hier wegnehmen, falls inkscape im Pfad ist
\begin{figure}
\centering
\includegraphics{svgexample.svg}
\caption{SVG direkt eingebunden}
\label{fig:directsvg}
\end{figure}

\begin{figure}
\centering
\def\svgwidth{.4\textwidth}
\includesvg{svgexample}
\caption{Text im SVG mittels \LaTeX{} gesetzt}
\label{fig:latexsvg}
\end{figure}
\fi % <-- Das hier wegnehmen, falls inkscape im Pfad ist

\section{Tabellen}
Tabelle~\ref{tab:Ergebnisse} zeigt Ergebnisse.
\begin{table}
  \begin{center}
    \begin{tabular}{ccc}
	\toprule
	\multicolumn{2}{c}{\textbf{zusammengefasst}} & \textbf{Titel} \\ \midrule
	Tabelle & wie & in \\
	\url{tabsatz.pdf}& empfohlen & gesetzt\\
	
	\multirow{2}{*}{Beispiel} & \multicolumn{2}{c}{ein schönes Beispiel}\\
	 & \multicolumn{2}{c}{für die Verwendung von \enquote{multirow}}\\
	\bottomrule
    \end{tabular}
    \caption[Beispieltabelle]{Beispieltabelle -- siehe \url{http://www.ctan.org/tex-archive/info/german/tabsatz/}}
    \label{tab:Ergebnisse}
  \end{center}
\end{table}

\section{Pseudocode}
Algorithmus~\vref{alg:sample} zeigt einen Beispielalgorithmus.
\begin{Algorithmus} %Die Umgebung nur benutzen, wenn man den Algorithmus ähnlich wie Graphiken von TeX platzieren lassen möchte
\label{alg:sample}
\caption{Sample algorithm}
\begin{algorithmic}
\Procedure{Sample}{$a$,$v_e$}
\State $\mathsf{parentHandled} \gets (a = \mathsf{process}) \lor \mathsf{visited}(a'), (a',c,a) \in \mathsf{HR}$ 
\State \Comment $(a',c'a) \in \mathsf{HR}$ denotes that $a'$ is the parent of $a$
\If{$\mathsf{parentHandled}\,\land(\mathcal{L}_\mathit{in}(a)=\emptyset\,\lor\,\forall l \in \mathcal{L}_\mathit{in}(a): \mathsf{visited}(l))$}
\State $\mathsf{visited}(a) \gets \text{true}$
\State $\mathsf{writes}_\circ(a,v_e) \gets
\begin{cases}
\mathsf{joinLinks}(a,v_e) & \abs{\mathcal{L}_\mathit{in}(a)} > 0\\
\mathsf{writes}_\circ(p,v_e)
& \exists p: (p,c,a) \in \mathsf{HR}\\
(\emptyset, \emptyset, \emptyset, false) & \text{otherwise}
\end{cases}
$
\If{$a\in\mathcal{A}_\mathit{basic}$}
  \State \Call{HandleBasicActivity}{$a$,$v_e$}
\ElsIf{$a\in\mathcal{A}_\mathit{flow}$}
  \State \Call{HandleFlow}{$a$,$v_e$}
\ElsIf{$a = \mathsf{process}$} \Comment Directly handle the contained activity
  \State \Call{HandleActivity}{$a'$,$v_e$}, $(a,\bot,a') \in \mathsf{HR}$
  \State $\mathsf{writes}_\bullet(a) \gets \mathsf{writes}_\bullet(a')$
\EndIf
\ForAll{$l \in \mathcal{L}_\mathit{out}(a)$}
  \State \Call{HandleLink}{$l$,$v_e$}
\EndFor
\EndIf
\EndProcedure
\end{algorithmic}
\end{Algorithmus}

\newpage
\section{Quellcode}
\lstinputlisting[language=C++,label=helloworld,caption={"`hello world"' in C++.}]{code/helloworld.cpp}


\section{Verweise}
Für weit entfernte Abschnitte ist \enquote{varioref} zu empfehlen:
\enquote{Siehe \vref{sec:mf}}.
Das Kommando \texttt{\textbackslash{}vref} funktioniert ähnlich wie \texttt{\textbackslash{}cref} mit dem Unterschied, dass zusätzlich ein Verweis auf die Seite hinzugefügt wird.
\texttt{vref}: \enquote{\vref{sec:diff}}, \texttt{cref}: \enquote{\cref{sec:diff}}, \texttt{ref}: \enquote{\ref{sec:diff}}.

Falls \enquote{varioref} Schwierigkeiten macht, dann kann man stattdessen \enquote{cref} verwenden.
Dies erzeugt auch das Wort \enquote{Abschnitt} automatisch: \cref{sec:mf}.
Das geht auch für Abbildungen usw.
Im Englischen bitte \verb1\Cref{...}1 (mit großen \enquote{C} am Anfang) verwenden.


%Mit MiKTeX Installation ab dem 2012-01-16 nicht mehr nötig
%Falls ein Abschnitt länger als eine Seite wird und man mittels \texttt{\textbackslash{}vref} auf eine konkrete Stelle in der Section
%verweisen möchte, dann sollte man \texttt{\textbackslash{}phantomsection} verwenden und dann wird
%auch bei \texttt{vref} die richtige Seite angeben.

%%The link location will be placed on the line below.
%%Tipp von http://en.wikibooks.org/wiki/LaTeX/Labels_and_Cross-referencing#The_hyperref_package_and_.5Cphantomsection
%\phantomsection
%\label{alabel}
%Das Beispiel für \texttt{\textbackslash{}phantomsection} bitte im \LaTeX{}-Quellcode anschauen.

%Hier das Beispiel: Siehe Abschnitt \vref{hack1} und Abschnitt \vref{hack2}.

\section{Definitionen}
\begin{definition}[Title]
\label{def:def1}
Definition Text
\end{definition}

\Cref{def:def1} zeigt \ldots

\section{Literaturverweise}
So wird auf eine Quelle~\cite{WSPA} verwiesen und so als Fußnote\footcite{WSPA}. Mit biblatex ist es auf möglich Zitate, z.B. aus dem Artikel von \textcite{WSPA}, direkt in den Textfluss zu integrieren.

\section{Verschiedenes}
\label{sec:diff}
\ifdeutsch
Ziffern (123\,654\,789) werden schön gesetzt.
Entweder in einer Linie oder als Minuskel-Ziffern.
Letzteres erreicht man durch den Parameter \texttt{osf} bei dem Paket \texttt{libertine} bzw.\ \texttt{mathpazo} in \texttt{fonts.tex}.
\fi

\textsc{Kapitälchen} werden schön gesperrt...

\begin{compactenum}[I]
\item Man kann auch die Nummerierung dank paralist kompakt halten
\item und auf eine andere Nummerierung umstellen
\end{compactenum}

\section{Weitere Informationen}
Verbesserungsvorschläge für diese Vorlage sind immer willkommen. Bitte bei github ein Ticket eingragen (\url{https://github.com/diehlpk/thesis-template/issues}).

\chapter{Zusammenfassung und Ausblick}\label{chap:zusfas}
Hier bitte einen kurzen Durchgang durch die Arbeit.

\section*{Ausblick}
...und anschließend einen Ausblick

\section*{Danksagung}
Dank an thesen@ins.uni-bonn.de und jneuen@ins.uni-bonn.de f\"ur die Beispiele zu tikz, pgfplots und das Einbinden von Sourcecode.


%
%
%\renewcommand{\appendixtocname}{Anhang}
%\renewcommand{\appendixname}{Anhang}
%\renewcommand{\appendixpagename}{Anhang}
\appendix
\input{content/anhang}
%\printindex
%\bibliographystyle{alphadin}
\ifdeutsch
\bibliographystyle{bibliography/IAASde} %f"ur deutsche Texte
\else
\bibliographystyle{bibliography/IAAS} %f"ur englische Texte
\fi
\bibliography{bibliography/bibliography}
\ifdeutsch
Alle URLs wurden zuletzt am 17.\,03.\,2008 geprüft.
\else
All links were last followed on March 17, 2008.
\fi

\backmatter 
\pagestyle{empty}
\renewcommand*{\chapterpagestyle}{empty}

%TODO: \Versicherung

\end{document}
